\textbf{intro}Physiological noise is one of the significant motion artefacts in functional magnetic resonance imaging(fMRI).
fMRI uses magnetic imaging to measure brain activity by measuring changes in local oxygenation of blood, which in turn reflects the amount of local brain activity.[1] However, physiological pulsations related to cardiac pulsation and respiration would perturb blood-oxygen level contrast(BOLD). First, respiration is inevitable. During data aquisition, even the perfect subject could not avoid chest movement from breathing, which would result in head motion in the magnetic field. So the MR images are likely to be distorted by respiration in a way. Second, cardiac pulsation might directly influence the BOLD signal especially in the region with large blood vessels. For example, cerebrospinal fluid(CSF) flow is modulated by both the cardiac and respiratory cycles, resulting in additional signal changes.[2] Moreover, due to the difference from the origins of the artefacts, the regions that they might appear could also be different.