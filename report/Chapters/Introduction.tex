Physiological noise is one of the significant artefacts in functional magnetic resonance imaging (fMRI).
fMRI uses magnetic imaging to measure brain activity by measuring changes in local oxygenation of blood, which in turn reflects the amount of local brain activity. \cite{poldrack2011handbook}
However, physiological pulsations related to cardiac pulsation and respiration would perturb blood-oxygen level contrast (BOLD) and add nuisance to fMRI data, which are the key challenges in this area. First, respiration is inevitable. 
During data aquisition, even the perfect subject could not avoid chest movement from breathing, which would result in head motion in the magnetic field. 
So that the MR images are likely to be distorted by respiration in a way.
Second, cardiac pulsation could directly influence the BOLD signal especially in the region with large blood vessels. 
For example, cerebrospinal fluid (CSF) flow is modulated by both the cardiac and respiratory cycles, resulting in additional signal changes.\cite{birn2012role}

The role of physiological noise in fMRI analysis deserves discussion. 
Some studies prefer not performing a physiological correction on fMRI data because cardiac and respiratory related fluctuations may be correlated with variations in neuronal activity.\cite{birn2012role} 
However, the spatial specificity of functional connectivity has been clearly demonstrated to be influenced by aliased physiological noise.\cite{lowe1998functional}
A functional connectivity study\cite{dagli1999localization} by Dagli has shown that removing cardiac fluctuations resulted in a variance reduction of roughly 10\%–40\%, 
depending on the region being investigated.\cite{dagli1999localization}
The sensitivity of these regions near pulsatile vessels could be improved by physiological noise correction. 
But another unarguable evidence is that heart rate variability is widely used as a measure of emotional arousal and autonomic nervous system activity.\cite{birn2012role} 
All these facts remind us that physiological noise correction shoule be used carefully especially when the key factors in experiments are related to physiological responses.

In this work, we provides a robust model-based tool XX in denoising physiological artefacts, which will give back a denoised data for further use. 
The implementation of this tool is based on the model RETROICOR.\cite{glover2000image}
This paper introduces the modules of this tool and explains results in each step from the example subject.