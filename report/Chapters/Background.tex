The goal of fMRI data analysis is to analyze each voxel's time series to see whether the BOLD signal changes in response to some manipulation.\cite{poldrack2011handbook} 
And the method used for fMRI analysis in detection of signal changes is the general linear model(GLM). 
GLM, a well-developed tool of linear systems analysis, is used widely to predict the responses of certain systems to a wide variety of inputs.\cite{cohen1997parametric}

In this work, our approach to addressing the problem of physiological artefacts is to record cardiac and respiratory signal during the scan, and then retrospectively remove these effects from the data.\cite{glover2000image} 
GLM model is used as the basis of the denoise tool to build a noise model representing physiological artefact in fMRI images, and the cleaned data is obtained by regressing out physiological components in the model.

The general linear model could be expressed as:
\begin{equation}
\label{eqn:glm}
    Y = X\beta + e
\end{equation}
In the expression, $X$ is a designed matrix containing linear variables for the model. And $Y$ is explained by the $X$ through weight $\beta$. And $\beta$ is the weight for each varaible. The error of this model is estimated by $e$.

Inheritating the idea of GLM, noise model RETROICOR assumes that the signal coule be expressed by additive noise components with weights. Both cardiac signal and respiration signal are regarded as periodic signals, so that they can be expressed as a low-Fourier series expanded interms of these phases:
\begin{multline}
    Y(t) = \sum_{m = 1}^{M} a_m^c cos(m\phi_c) + b_m^c sin(m\phi_c) + \\ a_m^r cos(m\phi_r) + b_m^r sin(m\phi_r)
\end{multline}

In this expression, $\phi_c$ and $\phi_t$ are the phases in the respective cardiac and respiratory cycles. 
And the order of Fouriour expression is configureable. In this work we used $M = 2$ as default.

For the calculation of cardiac phase, it is defined as
\begin{equation}
\label{eqn:cardiac}
    \phi_c(t) = 2\pi(t-t_1)/(t_2-t_1)
\end{equation}

Here $t_1$ and $t_2$ is the time for the subsequent R-peak value. 
So the expression constrains the phase value in the range of $0$ to $2\pi$. 
Since recorded physiological signals are used as inputs without any preprcossing, so R peak detection will be oprerated before the calculation.

For the calculation of respiratory phase, depth and states of breathing will be both taken into consideration. 
It is quite easy to imagine that a deeper breathe is more likely to bring a potential bulk motion of head. 
And also the phase of respiration is different from inhaling to exhaling. 
Therefore, respiratory phase is defined by both depth and state. 
State of inhaling or exhaling is more like polarity of phase, while depth is more like the amplitude value. 
The signal of depth is recorded by a detecting belt wrapping around subject's chest. 
To get the respiratory amplitude of the given time, a histogram-equalized transfer function is used, which is expressed  as :

\begin{equation}
    \phi_r(t) = \pi \frac{\sum_{b=1}^{rnd[R(t)/R_{max}]}H(b)}{\sum_{b=1}^{100}H(b)}dR/dt
\end{equation}

First, all the signal values are normalized to the range $(0, Rmax)$. Then a histogram $H(b)$ is gained with all normalized values. In the $y$ domain of histogram, the $y$ value is from the normalized data. 
And in the $x$ domain, the bins of histogram are defined to be 100 bins. 
If the normalized amplitude from the given time belongs to the $n$th bin, then the phase value will be calculated by dividing the sum value till $n$th bin by the sum of 100 bins.

In the end, value from the divisor will multiple by $\pi$, so that the phase value will be in the range of $-\pi$ to $\pi$.
During data aquisition of fMRI, BOLD time series have time resolution, which refers to reprtition time or TR. 
And by now MRI data will be either recorded by multi-slice or single slice at a time, which means the slices constructing the whole brain volumn come from different time stamp during the TR. 
Therefore, a single time stamp could not represent for the whole volume. 
However, applying different time stamps for each slice is also not practical. 
Because the slices are collected in horizontal direction, and the region of interest for artefacts could be in any shape rather than only scattering on a horizontal slice. 
So it is not meaningful to denoise slice by slice.
To simplify the slice time problem, the middle time stamp in TR is chosen to calculate the physiological phases.