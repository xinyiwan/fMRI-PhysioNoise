The validity of Physio Denoise need to be examined in the future work.
One way to execute this essential work is to compare the results with others from existing tools.
The preliminary idea is applying different physiologically denoising tools to the raw data.
Several typical ROIs are chosen to compare the predicted noise signal, raw signal and denoised signal.
The performance of different noise models, such as
the variance, Signal-to-noise ratio(SNR) and R-Squared, will be used as main criteria.

Another meaningful experiment is to remove the influence from respiration. 
Due to the correlation between respiration and motion artefact inside signal, removing respiratory regressors from the 
linear model will help Physio Denoise extract regions only focus on cardiac. Comparing the results with only cardiac Regressors
and with both regressors could give evidence in further study about the effect of respiration. 
Moreover, the data acquisition of cardiac signal during fMRI is very easy compared to respiration.
So this study may simplify the process with only cardiac signal, which will benefit users of our tool as SUBIC.

Last, independent component analysis (ICA) of functional MRI can also be used for results comparison.
Based on typical features of noises, ICA tools in fMRI automatically label noises. The performance of Physio Denoise will be 
easily evaluated with these sophisticated tools.